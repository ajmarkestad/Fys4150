%%%%
\documentclass[a4paper,11pt]{article}
%%%%
%%%%
%PACKAGES______________________________________________________________________________________
\usepackage{simplewick} %Allows Wick Notation
\usepackage{slashed} %Allows feynman slash notation 
\usepackage{graphicx} % graphics, pictures, figures
\usepackage{caption}
\usepackage{subcaption}
\usepackage{verbatim} % importing numerical scripts
\usepackage{multicol, float} % placing floats in right places
\usepackage{algpseudocode} % no idea...
\usepackage[utf8]{inputenc}
\usepackage{amssymb} %needed if not using mathdesign
\usepackage{amsmath}
\usepackage[OT1]{fontenc}
\usepackage{lmodern} %gfsartemisia, times, boisik, et cetera
\usepackage{braket} %dirac notation
\usepackage[cm]{fullpage} % for fulpage style
\usepackage{bm} % boldface vectors
\usepackage{float} % placing floats
\usepackage{relsize} % for \mathlarger command
\usepackage{mathrsfs} %?
\usepackage{textgreek} % cb-greek class
\usepackage{sectsty} % for centering sections
\usepackage{textcomp } % for nr. symbol
\usepackage[usenames, dvipsnames]{color} % defining own colors
\usepackage{type1cm} % scalable fonts
\usepackage{lettrine} % larger first letter in paragraph.
\usepackage{listings} % code-snippets in the text
\usepackage{background} % used for top page text
%\usepackage{niceframe} % for old-school double frame
\usepackage{tikz} % figure config/ creation
%\usepackage{bbold}
%\usepackage{swrule} % for fancy line
%\usepackage{pdfpages} % for importing pdf

%%%%
%%%% SET-UP NEEDED FOR FURTHER PACKAGES
%%%%
\definecolor{hyperclrblue}{RGB}{30,90,125} %Definind own color ; blue
\definecolor{hyperclrorng}{RGB}{210,100,45}%Definind own color
\definecolor{hyperclrgreen}{RGB}{60,120,20}%Definind own color
\usepackage[colorlinks = true,
linkcolor = hyperclrblue,
urlcolor = blue,
citecolor = blue,
anchorcolor = blue]{hyperref} % link package
\usepackage{pgfplots} % to plot directly into latex
\pgfplotsset{compat=1.5} % needed forpgfplots
\usepackage{framed, color} % for framing/shaded box
\definecolor{shadecolor}{cmyk}{0,0,0.185,0} % color for shaded box
\usepackage{fancybox}
\usepackage[sc]{titlesec} % title package
%_______________________________________________________________________________________________
%NEW COMMANDS_________________________________________________________________________________
%\renewcommand*{\thefootnote}{$\dagger$} % creating dagger footnote
\newcommand*{\boisik}{\fontfamily{bsk}\selectfont} % change font to boisik command
\newcommand{\wf}{\text{\textpsi}} % defining wavefunctions as cbgreek class.
\newcommand{\bwf}{\text{\textPsi}} % defining Wavefunctions as cbgreek class.
\newcommand{\Q}{\hat{\text{\boisik Q}}} % defining operator-style 'Q'
\newcommand{\nlm}{\ket{n\ell m_\ell}} % defining wavefunctions as cbgreek class.
\newcommand{\nlmz}{\ket{n\ell m_\ell;0}} % defining wavefunctions as cbgreek class.
\newcommand{\nlmt}{\ket{n\ell m_\ell;t}} % defining wavefunctions as cbgreek class.
%_____________________________________________
%\numberwithin{equation}{section} %equations labeled by section
\sectionfont{\centering} % centering sections with 'sectsty'
\subsectionfont{\centering} % centering sections with 'sectsty'
\definecolor{myclr}{RGB}{190,90,20} %Definind own color
\renewcommand{\thesection}{\Roman{section}.} % Roman numerals for sections
\renewcommand{\thesubsection}{\Alph{subsection}} % Roman numerals for subsections
\titleformat{\section}{\large\scshape\centering}{\thesection}{1em}{} % Change the look of the section titles
\titleformat{\subsection}{\normalsize\centering\bfseries}{\thesubsection.}{1em}{} % Change the look of the section titles
\setlength{\columnsep}{0.7cm}
%______________________________________________________________________________________________
%%%%
%%%%_________________________________________________________________________________________
\begin{document}
%%%% TOP PAGE TEXT
{\SetBgContents{ \textit{{\small\textsc{ Ask J. Markestad, Thorbjørn V. Larsen Universitetet i Oslo. \hspace{3.5cm} \textit{\today}}}}}
\SetBgScale{1}
\SetBgColor{black}
\SetBgAngle{0}
\SetBgOpacity{1}
\SetBgPosition{current page.north east}
\SetBgVshift{-1.2cm}
\SetBgHshift{-10.5cm}
%%%% CREATING TITLE HEADER
$$\:$$
\begin{center}
	\vspace{0.2cm}%\boisik
	\fontsize{15}{15}\selectfont \textsc{ Project 2:Jacobis method},\\
	%{in}}\\
	\fontsize{13}{13}\selectfont \textsc{Fys $\textnormal{{4150}}$ }\\
	\vspace{0.4cm}
	\fontsize{12}{12}\selectfont {\textsc{ Ask J. Markestad, Thorbjørn V. Larsen }}\\
	\vspace{0.5cm}
\end{center}
%%%%
%%%%
%______________________________________________________________________________________________
%%%%
%%%%
	
%\includegraphics[scale = 0.48]{line}
\rule{\textwidth}{0.3pt}\par
		
%---------------------------------------------------------------------------------------------------------------------------------------
\begin{abstract}
	Jacobis method
\end{abstract}



		
\section*{Introduction}
		
		
		
		
\section*{Theory and Algorithms}
		\begin{equation}
\begin{pmatrix}
	2 & -1 & 0 & ... & ... & 0 \\
	-1 & 2 & -1 & 0 & ... & 0 \\
	0 & -1 & 2 & -1 & 0 & ... \\
	... & ... & ... & ... & ... & ... \\
	0 & ... & 0 & -1 & 2 & -1 \\
	0 & 0 & ... & 0 & -1 & 2 
	\end{pmatrix} \begin{pmatrix}
	u_1\\
	u_2\\
	u_3\\
	...\\
	...\\
	u_n
	\end{pmatrix} = \begin{pmatrix}
	f_1 h^2 \\
	f_2 h^2 \\
	f_3 h^2 \\
	... \\
	... \\
	f_n h^2 \\
	\end{pmatrix}
\end{equation}		

\subsection{Memory handling and algorithms}

Source code and accompanying codes can be found at the git hub address:

\url{https://github.com/ajmarkestad/Fys4150/tree/master/Project2} 


\begin{lstlisting}
//general forward algorithm
for (int i=1; i<=n; i++)
	{
	b[i]=b[i]-c[i-1]*b[i-1]/b[i-1];
	f[i]=f[i]-c[i-1]*f[i-1]/b[i-1];
	}
\end{lstlisting}

\section*{Results}
\section*{Conclusion}
 
		
		
		
		
\begin{thebibliography}{3}
			
	\bibitem{M.Hjort-Jensen_CompFys}
	Morten Hjort-Jensen
	\emph{ Computational Physics Lecture Notes Fall 2015}
	Department of Physics, University of Oslo
	2015
	\url{https://github.com/CompPhysics/ComputationalPhysics/blob/master/doc/Lectures/lectures2015.pdf}
			
			
			
			
			
\end{thebibliography}
		
		
		
		
		
		
		
		
		
%__________________________________________________________________________
\end{document}