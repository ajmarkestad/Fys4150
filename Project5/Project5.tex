%%%%
\documentclass[a4paper,11pt]{article}
%%%%
%%%%
%PACKAGES______________________________________________________________________________________
\usepackage{simplewick} %Allows Wick Notation
\usepackage{slashed} %Allows feynman slash notation 
\usepackage{graphicx} % graphics, pictures, figures
\usepackage{caption}
\usepackage{subcaption}
\usepackage{verbatim} % importing numerical scripts
\usepackage{multicol, float} % placing floats in right places
\usepackage{algpseudocode} % no idea...
\usepackage[utf8]{inputenc}
\usepackage{amssymb} %needed if not using mathdesign
\usepackage{amsmath}
\usepackage[OT1]{fontenc}
\usepackage{lmodern} %gfsartemisia, times, boisik, et cetera
\usepackage{braket} %dirac notation
\usepackage[cm]{fullpage} % for fulpage style
\usepackage{bm} % boldface vectors
\usepackage{float} % placing floats
\usepackage{relsize} % for \mathlarger command
\usepackage{mathrsfs} %?
\usepackage{textgreek} % cb-greek class
\usepackage{sectsty} % for centering sections
\usepackage{textcomp } % for nr. symbol
\usepackage[usenames, dvipsnames]{color} % defining own colors
\usepackage{type1cm} % scalable fonts
\usepackage{lettrine} % larger first letter in paragraph.
\usepackage{listings}
\usepackage{background} % used for top page text
%\usepackage{niceframe} % for old-school double frame
\usepackage{tikz} % figure config/ creation
%\usepackage{bbold}
%\usepackage{swrule} % for fancy line
%\usepackage{pdfpages} % for importing pdf

%%%%
%%%% SET-UP NEEDED FOR FURTHER PACKAGES
%%%%
\definecolor{hyperclrblue}{RGB}{30,90,125} %Definind own color ; blue
\definecolor{hyperclrorng}{RGB}{210,100,45}%Definind own color
\definecolor{hyperclrgreen}{RGB}{60,120,20}%Definind own color
\usepackage[colorlinks = true,
linkcolor = hyperclrblue,
urlcolor = blue,
citecolor = blue,
anchorcolor = blue]{hyperref} % link package
\usepackage{pgfplots} % to plot directly into latex
\pgfplotsset{compat=1.5} % needed forpgfplots
\usepackage{framed, color} % for framing/shaded box
\definecolor{shadecolor}{cmyk}{0,0,0.185,0} % color for shaded box
\usepackage{fancybox}
\usepackage[sc]{titlesec} % title package
%_______________________________________________________________________________________________
%NEW COMMANDS_________________________________________________________________________________
%\renewcommand*{\thefootnote}{$\dagger$} % creating dagger footnote
\newcommand*{\boisik}{\fontfamily{bsk}\selectfont} % change font to boisik command
\newcommand{\wf}{\text{\textpsi}} % defining wavefunctions as cbgreek class.
\newcommand{\bwf}{\text{\textPsi}} % defining Wavefunctions as cbgreek class.
\newcommand{\Q}{\hat{\text{\boisik Q}}} % defining operator-style 'Q'
\newcommand{\nlm}{\ket{n\ell m_\ell}} % defining wavefunctions as cbgreek class.
\newcommand{\nlmz}{\ket{n\ell m_\ell;0}} % defining wavefunctions as cbgreek class.
\newcommand{\nlmt}{\ket{n\ell m_\ell;t}} % defining wavefunctions as cbgreek class.
%_____________________________________________
%\numberwithin{equation}{section} %equations labeled by section
\sectionfont{\centering} % centering sections with 'sectsty'
\subsectionfont{\centering} % centering sections with 'sectsty'
\definecolor{myclr}{RGB}{190,90,20} %Definind own color
\renewcommand{\thesection}{\Roman{section}.} % Roman numerals for sections
\renewcommand{\thesubsection}{\Alph{subsection}} % Roman numerals for subsections
\titleformat{\section}{\large\scshape\centering}{\thesection}{1em}{} % Change the look of the section titles
\titleformat{\subsection}{\normalsize\centering\bfseries}{\thesubsection.}{1em}{} % Change the look of the section titles
\setlength{\columnsep}{0.7cm}
%______________________________________________________________________________________________
%%%%
%%%%_________________________________________________________________________________________
\begin{document}
%%%% TOP PAGE TEXT
{\SetBgContents{ \textit{{\small\textsc{ Ask J. Markestad, Thorbjørn V. Larsen Universitetet i Oslo. \hspace{3.5cm} \textit{\today}}}}}
\SetBgScale{1}
\SetBgColor{black}
\SetBgAngle{0}
\SetBgOpacity{1}
\SetBgPosition{current page.north east}
\SetBgVshift{-1.2cm}
\SetBgHshift{-10.5cm}
%%%% CREATING TITLE HEADER
$$\:$$
\begin{center}
	\vspace{0.2cm}%\boisik
	\fontsize{15}{15}\selectfont \textsc{ Project 5: Monte Carlo simulation of financial interactions between homogeneous agents}\\
	%{in}}\\
	\fontsize{13}{13}\selectfont \textsc{Fys $\textnormal{{4150}}$ }\\
	\vspace{0.4cm}
	\fontsize{12}{12}\selectfont {\textsc{ Ask J. Markestad, Thorbjørn V. Larsen }}\\
	\vspace{0.5cm}
\end{center}
%%%%
%%%%
%______________________________________________________________________________________________
%%%%
%%%%
	
%\includegraphics[scale = 0.48]{line}
\rule{\textwidth}{0.3pt}\par
		
%---------------------------------------------------------------------------------------------------------------------------------------
\begin{abstract}
	In this paper we investigate the interactions of 
\end{abstract}



		
\section*{Introduction}







\section*{Theory and Algorithms}



\subsection{Probability and normalization}

\subsection*{Algorithms}

\begin{lstlisting}
for all timesteps:
	for all bodies:
		x_(i+1) = f(a_i,x_i,v_i)
	for all bodies:
		a_i+1 = f'(x_i+1)
	for all bodies:
		v_(i+1)=f''(v_i, a_i+1, a_i)
	for all bodies:
		a_i = a_i+1
\end{lstlisting}Our Github page for the calculations, program files and bechmarks is found on 

\url{https://github.com/ajmarkestad/Fys4150/tree/master/Project3}


\section*{Results}
\verbatiminput{saving.txt}

\subsection{5a}
\begin{figure}[H]
	\centering
	\includegraphics[scale=0.5]{testinit}
	\caption{Variance as the system approaches the equilibrium. We see that the system quickly goes to the equilibrium situation after approximately 10000 transactions.  1000 agents, 10 transactions per run, $\gamma=\alpha=\lambda = 0$,  20000 bins  }
	\label{fig:testinit}
\end{figure}

\subsection{5b}




\begin{figure}[H]
	\centering
	\begin{subfigure}[t]{0.45\textwidth}
		\includegraphics[scale=0.4]{propersimpleinit}
		\caption{Variance of the initial 100 runs, where we see that we get to the equilibrium after the first few run.}
		\label{fig:propersimpleinit}
	\end{subfigure}
	\begin{subfigure}[t]{0.45\textwidth}
		\includegraphics[scale=0.4]{Proper_simple_transaction_log}
		\caption{Loglog plot of the distribution of money. }
		\label{fig:Proper_simple_transaction_log}
	\end{subfigure}
	\caption{Simple transaction, with 500 agens, 2000 total runs, $2\cdot 10^{6}$ transactions per run and  $\lambda=0$, $\alpha=0$, $\gamma=0$, 20000 bins, 100 initial runs}
	\label{fig:simple}
\end{figure}


\subsection{5c. Lambda models}


\begin{figure}[H]
	\centering
	\begin{subfigure}[t]{0.45\textwidth}
		\includegraphics[scale=0.4]{savinginit}
		\caption{Variance of the initial 100 runs, where we see that we get to the equilibrium before we start sampling data}
		\label{fig:savinginit}
	\end{subfigure}
	\begin{subfigure}[t]{0.45\textwidth}
		\includegraphics[scale=0.4]{Saving}
		\caption{Introducing a savings amount $\lambda$ affects the wealth distribution of the agens to a large degree where we see that with a higher savings fraction the distribution becomes to a larger degree gaussian}
		\label{fig:Saving}
	\end{subfigure}
	\caption{Savings models, with 500 agents, 10000 total runs, $2\cdot 10^{6}$ transactions per run and  $\lambda=[0,0.25,0.9]$, $\alpha=0$, $\gamma=0$,20000 bins}
	\label{fig:savings}
\end{figure}

We use a scipy.curve_fit to fit the observed data to an exponential of the form
\begin{align}
	count (money) = Ax^{-1-\nu} + C
\end{align}
which is equal to the equation in REFERANSE. 

\begin{table}[h]
\caption{default}
\begin{center}
\begin{tabular}{|c|c|}
$\lambda$ & 0 & 0.25 & 0.9 \\
$\nu$ & 
\end{tabular}
\end{center}
\label{default}
\end{table}%



\subsection{5d.Nearest neighbor interactions}








\begin{figure}[H]
	\centering
	\begin{subfigure}[b]{0.27\textwidth}
		\includegraphics[scale=0.27]{propersimpleinit}
		\caption{Variance of the initial 40000 transactions, where we see that we get to the equilibrium. }
		\label{fig:propersimpleinit}
	\end{subfigure}
	\begin{subfigure}[b]{0.27\textwidth}
		\includegraphics[scale=0.27]{Proper_simple_transaction_log}
		\caption{Logarithm of the histogram. }
		\label{fig:Proper_simple_transaction_log}
	\end{subfigure}
	\begin{subfigure}[b]{0.27\textwidth}
		\includegraphics[scale=0.27]{Proper_simple_transaction_log}
		\caption{Logarithm of the histogram. }
		\label{fig:Proper_simple_transaction_log}
	\end{subfigure}
	\caption{Simple transaction, $\lambda=0$, $\alpha=0$, $\gamma=0$}
	\label{fig:simple}
\end{figure}




\section*{Conclusion}





\begin{thebibliography}{3}
			
	\bibitem{M.Hjort-Jensen_CompFys}
	Morten Hjort-Jensen
	\emph{ Computational Physics Lecture Notes Fall 2015}
	Department of Physics, University of Oslo
	2015
	\url{https://github.com/CompPhysics/ComputationalPhysics/blob/master/doc/Lectures/lectures2015.pdf}
	
	
			
			
			
\end{thebibliography}
		
		
		
		
		
		
		
		
		
%__________________________________________________________________________
\end{document}